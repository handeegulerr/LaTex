\documentclass{article}
\usepackage[utf8]{inputenc}
\usepackage{amsmath}
\usepackage[top=1in]{geometry}
\usepackage{hyperref}
\usepackage{color}
\usepackage{physics}
\usepackage{mathtools}
\usepackage{chemformula}
\usepackage{mhchem}

\setlength{\parindent}{1em}
\setlength{\parskip}{1em}
\renewcommand{\baselinestretch}{1.0}


\title{Hello World!}
\author{Hande Güler}
\date{April 7, 2021}

\begin{document}

\maketitle

\section{Getting Started}

\textbf{Hello World!}  Today I am learning \LaTeX.\LaTeX\ is a great program for writing math. I can write in line math such as $a^2 + b^2 = c^2$. I can also give equations their own space:

\begin{align}
    \gamma^2 + \theta^2 = \omega^2 \\[\baselineskip]
    \nonumber
\end{align}
\text{"Maxwell's equantions" are named for James Clark Maxwell and are as follow:}\\
\begin{align}
    \Vec{\nabla}\cdot\Vec{E} \quad& = \quad \frac{\rho}{\epsilon_0}&&\text{Gauss's Law} \\
    \Vec{\nabla}\cdot\Vec{B}\quad& =\quad 0 &&  \text{Gauss's Law of Magnetism} \\
    \Vec{\nabla} \times \Vec{E}\quad& =\quad -\frac{\partial\Vec{B}}{\partial{t}} && \text{Faraday's Law of Induction} \\
    \Vec{\nabla} \times  \Vec{B}\quad& = \quad \mu_0    \bigg(\epsilon_0\frac{\partial\Vec{E}}{\partial{t}} + \Vec{J}\bigg) && \text{Ampere's Circuital Law}
\end{align}    
\text{Equations {\color{blue}2}, {\color{blue}3}, {\color{blue}4} and {\color{blue}5} some of the most important in Physics.} 

\section{What about Matrix Equations?}
\begin{center}

\[
\begin{pmatrix}
a_{11} & a_{12} & \cdots & a_{1n} \\
a_{21} & a_{22} & \cdots & a_{1n} \\
\vdots & \vdots & \ddots & \vdots \\
a_{n1} & a_{n2} & \cdots & a_{nn} \\
\end{pmatrix}
\left[\begin{array}{@{}c@{}}
    v_{1} \\
    v_{2} \\
    \vdots \\
    v_{n} 
    \end{array} \right]
= 
\begin{matrix}
w_{1} \\
w_{2} \\
\vdots \\
w_{n}
\end{matrix} 
\]
\newpage
\[
\displaystyle \iiint\limits_V f(x,y,z) \, \mathrm{d}V = F
\]
\[\displaystyle\frac{\mathrm{d}x}{\mathrm{d}y} \, = x'\; = \lim_{h\to \,0} \frac{f(x+h)-f(x)}{h}
\]

\[
\mid x \mid = 
\begin{cases}
\begin{align}
-x, \ if\ x < 0\\
x, \ if\ x \geq 0
\end{align}
\end{cases}  
\]

\[
F(x) = A_0 + \sum_{n=1}^{N}
\left[\Bigg\ 
A_n\cos\left( {\frac{2\pi nx}{P}}\right) 
+ 
B_n \sin\left( {\frac{2\pi nx}{P}}\right) \right\Bigg] 
\]
\[
\displaystyle \sum_{n}\frac{1}{n^s} = \prod_{p}\frac{1}{1-\frac{1}{p^s}}
\]
\[
\displaystyle\ m\ddot{{x}} + c\dot{{x}} + kx = F_0\sin\left( {2\pi ft}\right)
\]
\begin{align*}
f(x)\quad&=\quad x^2 + 3x + 5x^2 + 8 + 6x\\
\quad&=\quad6x^2 + 9x + 8\\
\quad&=\quad x\left (6x + 9\right) + 8
\end{align*}

\[
X\quad=\quad\frac{F_0}{k} \ \frac{1}{\sqrt{\left(1-r^2\right)^2 + \left(2\zeta r\right)^2}}
\]

\[
G_{\mu\nu} \equiv R_{\mu\nu}  - \frac{1}{2}Rg_{\mu\nu}  = \frac{8\pi G}{c^4}T_{\mu\nu} 
\]

\[
\ce {6CO2 + 6H2O \rightarrow C6H12 + 6O2}\\
\ce{SO4^2- + Ba^2 \rightarrow BaSO4}
\]

\[
\begin{pmatrix}
a_{11} & a_{12} & \cdots & a_{1n} \\
a_{21} & a_{22} & \cdots & a_{1n} \\
\vdots & \vdots & \ddots & \vdots \\
a_{n1} & a_{n2} & \cdots & a_{nn} \\
\end{pmatrix}
\left(\begin{array}{@{}c@{}}
    v_{1} \\
    v_{2} \\
    \vdots \\
    v_{n} 
    \end{array} \right)
= 
\left(\begin{matrix}
w_{1} \\
w_{2} \\
\vdots \\
w_{n}
\end{matrix} \right)
\]

\[
\frac{\partial\textbf{u}}{\partial} + \left(\textbf{u}  \cdot \nabla \right)\textbf{u}  - \nu \nabla ^2  \left(\textbf{u} \right) = - \nabla\textbf{h}
\]
\[\alpha A \beta B \gamma \Gamma \delta\Delta\pi	\Pi\omega\Omega\]
\end{center}
\end{document}




\begin{document}
\begin{center}

\end{center}
\end{document}